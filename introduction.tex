\chapter{Introduction}
\label{sec:Introduction}

\section{Parallel Architectures and Parallel Programming Models}
\label{sec:Introduction}

Parallel computing is a type of computation in which calculations are carried out simultaneously~\cite{Almasi:1989:HPC:160438}. The idea is to divide large problems into smaller ones and solve those smaller problems at the same time.

Flynn's taxonomy is a classification of computer architectures, proposed by Michael J. Flynn in 1966~\cite{5009071,44900}. This classification is based on the number of concurrent instructions and data streams available in the architecture.
There are four categories defined:

\bf{Single Instruction, Single Data Stream (SISD)}

\bf{Single Instruction, Multiple Data Streams (SIMD)}

\bf{Multiple Instruction, Single Data Stream (MISD)}

\bf{Single Instruction, Multiple Data Streams (SIMD)}


\section{Message Passing Interface (MPI)}
\label{sec:Introduction}


\section{Influence of Runtime Environments on MPI Jobs}
\label{sec:Introduction}


\section{Challenges for Runtime Environments}
\label{sec:Introduction}


\section{Organization of This Document}
\label{sec:Introduction}

