High-performance computing systems are growing toward hundreds-of-thousands to million-node machines, utilizing the computing power of billions of cores. Running parallel applications on such large machines efficiently will require optimized runtime environments that are scalable and resilient. Multi- and many-core chip architectures in large-scale supercomputers pose several new challenges to designers of operating systems and runtime environments.

ParalleX is a general-purpose parallel-execution model aiming to overcome the limitations imposed by the current hardware and the way we write applications today. High-Performance ParalleX (HPX) is an experimental runtime system for ParalleX.

The majority of scientific and commercial applications in HPC are written in MPI. In order to facilitate the transition from MPI model to ParalleX, there is a need for a compatibility mechanism between the two. Currently, this mechanism does not exist. This thesis provides a compatibility mechanism for MPI applications to use the HPX runtime system. This is achieved by developing a new runtime system for the Open MPI project, an open source implementation of MPI. This new runtime system is called HPX-RTE.

HPX-RTE is a new, lightweight, and open-source runtime system specifically designed for the emerging exascale computing environment. The system is designed relying on HPX project advanced features to allow for easy extension and transparent scalability. HPX-RTE provides full compatibility for current MPI applications to run on HPX runtime system. HPX-RTE provides an easy and simple path for transition from MPI to HPX. It also paves the way for future hybrid programming models such as HPX-MPI and integration of more features from HPX into Open MPI.
